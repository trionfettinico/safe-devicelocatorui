\documentclass[a4paper]{article}
\usepackage[T1]{fontenc}
\usepackage[utf8]{inputenc}
\usepackage[italian]{babel}
\usepackage{enumitem}
\begin{document}

\author{Lorenzo Tanganelli \and Luca Patarca \and Nico Trionfetti}
\title{Safe-Device Locator UI}
\maketitle

\newpage
\tableofcontents

\newpage

\section*{{Introduzione}}
Questo progetto fa parte del progetto di ricerca industriale S.A.F.E che ha come obbiettivo la realizzazione di sistemi di arredo innovativi capaci di trasformarsi in sistemi inteligenti di protezione passiva delle persone in caso di crollo dell'edificio causato da un terremoto.

Questi sistemi di arredo smart saranno dotati di sensoristica "salva-vita" capace di pre-allertare in caso di terremoto, di rilevare e localizzare la presenza di vita dopo un crollo, di monitorare le condizioni ambientali sotto le macerie e di elaborare e trasmettere informazioni utili a chi deve portare soccorso.

Il ciclo di vita dei sensori si divide in tre scenari operativi:
\begin{enumerate}[label=\roman{*}., ref=(\roman{*})]
    \item \textbf{Tempo di pace:} monitoraggio per il pre-allertamento (es. misure accellerometriche)
    \item \textbf{Durante l'evento:} invio dei dati per il rilevamento dei danni (es. misure accellerometriche, inclinometriche e di spostamento) e attivazione di logiche di intervento in seguito al riconoscimento dell'evento.
    \item \textbf{Dopo l'evento:} invio dei dati per la localizzazione delle vittime e monitoraggio ambientale al fine di guidare gli operatori nel triage di soccorso.
    \end{enumerate}

L'invio di dati tra i sensori ed il mondo esterno avviene utilizzando la tecnologia LoRa.\newline
LoRa consente trasmissioni a lungo raggio e a basso consumo energetico arrivando oltre 10 km nelle zone rurali e 3–5 km in zone fortemente urbanizzate.

Facendo riferimento al modello ISO/OSI la tecnologia è presente in due strati: 
\begin{itemize}
    \item \textbf{LoRa: } Il livello fisico LoRa è proprietario della Semtech e non se ne conoscono i dettagli implementativi.
    LoRa utilizza una modulazione a spettro espanso proprietaria derivato della modulazione Chirp Spread Spectrum (CSS). Inoltre utilizza la codifica Forward Error Correction (FEC) come meccanismo di rilevazione e successiva correzione degli errori contro le interferenze. 
    \item \textbf{LoRaWAN: } LoRaWAN è un protocollo del livello Media Access Control (MAC) che lavora a livello di rete per la gestione delle comunicazioni tra gateway Low Power Wide Area Network (LPWAN) e dispositivi end-node come protocollo di routing.
    \end{itemize}

Lo scenario operativo post evento si divide in tre attività:
\begin{enumerate}[label=\roman{*}., ref=(\roman{*})]
    \item \textbf{Campionatura:} mediante l'utilizzo di un drone dotato di tecnologia che supporta il protocollo LoRaWAN viene campionata l'area coperta dalle macerie. Durante la fase di volo vengono memorizzati i dati ricevuti dai sensori e la potenza del segnale.
    \item \textbf{Analisi dati:} sfruttando opportuni algoritmi di localizzazione vengono analizzati i dati memorizzati dal drone così da determinare dei centroidi in cui si suppone si trovi il disperso. 
    \item \textbf{Guidare soccorittori:} i soccorittori, dotati di opportuni tablet, visualizzerano una mappa con la heatmap e i centroidi risultanti dall'attività di analisi dati
    \end{enumerate}

Il nostro progetto ha l'obbiettivo di creare un applicativo per tablet linux e android per l'ultima attività post evento.\newline 
I requisiti funzionali del sistema 

\end{document}